\chapter*{Analyse des risques}
\addcontentsline{toc}{chapter}{Analyse des risques}
\label{sec:risques}
\section*{Gestion des risques}
\addcontentsline{toc}{section}{Gestion des risques}
La gestion des risques est un point clé dans le succès du projet car elle permet d'agir de manière à ce que les risques identifiés arrivent avec la probabilité la plus faible possible. En concertation avec les membres du projet, nous avons identifié les risques suivant:

\begin{itemize}
	\item Perte des données. Une panne matériel peut survenir à tout moment et endommager les données. Ce risque reste assez faible et on peut toujours récupérer les données en les téléchargeant une nouvelle fois.
	
	\item Difficulté d'accès aux données. Si les données sont stockées sur un ordinateur personnel, il y a un risque que cela ralentisse le développement des codes de calculs. C'est pourquoi nous avons envisagé de stocker les données sur un disque accessible depuis n'importe quel ordinateur de l'école.
	
	\item Logiciel inconnu par certain membre du groupe. Nous devrons prendre en compte le fait que tous les membres ne connaissent pas les librairies utilisés et donc évaluer le temps de travail en fonction de cela.
	
	\item Crise du coronavirus. Dans le contexte actuel, il est très probable qu'un des membres tombent malade ou que la communication entre les membres soient plus difficile. Nous devons donc faire en sorte que la majorité des tâches puissent se faire depuis un ordinateur personnel.
	
	\item Mauvaise communication interne. La plupart du travail sera faite en dehors des crénaux prévus pour le PIE. Une mauvaise communication pouvant entrainer retarder le retarder l'exécution du projet, nous allons mettre en place un groupe de conversation messenger ainsi qu'un drive Google où seront déposés les documents relatifs à la gestion du projet.
	
\end{itemize}

\section*{Opportunités}
\addcontentsline{toc}{section}{Opportunités}
%%% Local Variables: 
%%% mode: latex
%%% TeX-master: "isae-report-template"
%%% End: 

