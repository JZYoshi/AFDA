\chapter*{Organisation du projet}
\addcontentsline{toc}{chapter}{Organisation du projet}
\markboth{Organisation du projet}{Organisation du projet}
\label{sec:organisation}


\section*{Work Breakdown Structure et Planning}
\addcontentsline{toc}{section}{Work Breakdown Structure}

Afin que chacun puisse s'investir dans le projet, il était nécessaire de le diviser en tâches simples et de bien cerner l'ensemble des choses à faire. Le WBS en figure \ref{fig:wbs} présente ces principales tâches, qui seront réparties à chacun des membres du projet dans la section suivante. 

\begin{figure}[!h]
	\centering
	\includegraphics[width=12cm]{WBS}
	\caption{WBS}
	\label{fig:wbs}
\end{figure}


Par ailleurs, il était important d'agencer ces travaux dans le temps afin de bien respecter les délais, ainsi que de voir ce qui pouvait être fait en parallèle. Les figures \ref{fig:mvp-plan}, \ref{fig:gestion-proj} et \ref{fig:r1-plan} montrent le planning que nous avons construit et suivi au long de l'année.


\begin{figure}[!ht]
	\centering
	\includegraphics[width=15cm]{MVP planning}
	\caption{planning associé au produit minimal(MVP)}
	\label{fig:mvp-plan}
\end{figure}

\begin{figure}[!ht]
	\centering
	\includegraphics[width=15cm]{R1 planning}
	\caption{planning associé au produit final}
	\label{fig:r1-plan}
\end{figure}

\begin{figure}[!ht]
	\centering
	\includegraphics[width=15cm]{gestion projet}
	\caption{planning associé à la gestion de projet}
	\label{fig:gestion-proj}
\end{figure}

\section*{Association des taches aux membres du projet et Organigramme}
\addcontentsline{toc}{section}{Association des taches aux membres du projet}

Pour la répartition des travaux, un organigramme ainsi qu'une matrice d'attribution des tâches à chaque membre du projet ont été effectués et présentés en figure \ref{fig:Organigramme} et \ref{fig:raci}.

\begin{figure}[!ht]
	\centering
	\includegraphics[width=10cm]{Organigramme}
	\caption{Organigramme}
	\label{fig:Organigramme}
\end{figure}

\begin{figure}[!h]
	\centering
	\includegraphics[width=15cm]{RACI}
	\caption{matrice RACI, R:responsible, A:accountable, C:consulted, I:informed}
	\label{fig:raci}
\end{figure}



\section*{Suivi}
\addcontentsline{toc}{section}{Suivi}
Afin que la communication au sein du projet soit de bonne qualité, nous avons mis en place un drive google où sont stockés les documents qui concernent la gestion du projet (cahier charge, compte rendu de réunion,...). Par ailleurs, nous avons mis en place un répertoire github pour assurer les travaux de développement et de rédaction des documents qui sont fait en équipe. Ces deux répertoires sont accessibles et modifiables depuis n'importe quel poste et par n'importe quel membre du projet. 

Chaque semaine, une réunion a été organisé ce qui a permis de faire un point régulier de l'avancement du projet ainsi que sur la répartition des tâches restantes au sein de l'équipe. 

D'autre part, nous avons utilisé la librairie pandas sous python pour assurer le traitement des données, ainsi que django pour réaliser une  applications web. 



\section*{Gestion des risques et Opportunités}
\addcontentsline{toc}{section}{Gestion des risques}
La gestion des risques est un point clé dans le succès du projet car elle permet d'agir de manière à ce que les risques identifiés arrivent avec la probabilité la plus faible possible. En concertation avec les membres du projet, nous avions identifié les risques suivant:

\begin{itemize}
	\item Perte des données. Une panne matériel peut survenir à tout moment et endommager les données. Ce risque reste assez faible et on peut toujours récupérer les données en les téléchargeant une nouvelle fois.
	
	\item Difficulté d'accès aux données. Si les données sont stockées sur un ordinateur personnel, il y a un risque que cela ralentisse le développement des codes de calculs. C'est pourquoi nous avons envisagé de stocker les données sur un disque accessible depuis n'importe quel ordinateur de l'école.
	
	\item Logiciel inconnu par certain membre du groupe. Nous avons du  prendre en compte le fait que tous les membres ne connaissent pas les librairies utilisées et donc évaluer le temps de travail en fonction de cela.
	
	\item Crise du coronavirus. Dans le contexte actuel, il est très probable qu'un des membres tombent malade ou que la communication entre les membres soient plus difficile. Nous devons donc faire en sorte que la majorité des tâches puissent se faire depuis un ordinateur personnel.
	
	\item Mauvaise communication interne. La plupart du travail sera faite en dehors des créneaux prévus pour le PIE. Une mauvaise communication pouvant entrainer un retard dans l'exécution du projet, nous avons mis en place un groupe de conversation messenger ainsi qu'un drive Google où sont déposés les documents relatifs à la gestion du projet.
	
\end{itemize}


Nous avons listé ci-dessous des opportunités qui peuvent nous permettre d'améliorer le développement du projet.

\begin{itemize}
	\item Matériel informatique de l'école. L'école dispose de ressources informatiques importantes qui peuvent nous permettre d'effectuer des calculs rapidement et de stocker une quantité importante de données. Afin de profiter de cela, nous avons fait une demande de matériel à l'école.
	
	\item La librairie Openap permet de calculer les différentes phases d'un vol. En parvenant à l'utiliser, on peut gagner beaucoup sur cette tâche du projet. Elle est de plus très bien documentée, son utilisation sera donc probablement assez simple.
\end{itemize}
%%% Local Variables: 
%%% mode: latex
%%% TeX-master: "isae-report-template"
%%% End: 